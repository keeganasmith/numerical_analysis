\documentclass{article}
\usepackage{fancyhdr}
\usepackage{lipsum}  
\usepackage{listings} 
\usepackage{xcolor}   
\usepackage{amsmath}
\usepackage{enumitem}
\usepackage{graphicx}
% Define macros for title and author
\newcommand{\thetitle}{MATH 417 502 \\ Homework 4}
\newcommand{\theauthor}{Keegan Smith}

\title{\thetitle}
\author{\theauthor}

\pagestyle{fancy}
\fancyhf{}  % Clear all header and footer fields
\fancyhead[L]{\nouppercase{\rightmark}}
\fancyhead[C]{\thetitle}  % Title in the center
\fancyhead[R]{\theauthor}  % Your name on the right

\lstset{ %
  backgroundcolor=\color{white},   % choose the background color
  basicstyle=\ttfamily\small,          % size of fonts used for the code
  keywordstyle=\color{black},           % color for keywords
  commentstyle=\color{black},          % color for comments
  stringstyle=\color{black},             % color for strings
  numbers=left,                        % where to put the line-numbers
  numberstyle=\tiny\color{gray},       % style for line-numbers
  stepnumber=1,                        % the step between two line-numbers
  numbersep=5pt,                       % how far the line-numbers are from the code
  frame=single,                        % adds a frame around the code
  rulecolor=\color{black},             % frame color
  breaklines=true,                     % automatic line breaking
  breakatwhitespace=false,             % automatic breaks should only happen at whitespace
  showspaces=false,                    % don't show spaces in the code
  showstringspaces=false,              % don't show spaces in strings
  showtabs=false,                      % don't show tabs in the code
}

\begin{document}
\maketitle
\section*{Problem 1}
\begin{figure}[htbp]
  \centering
  \includegraphics[width=\linewidth]{images/f\_5.png}
\end{figure}

\begin{figure}[htbp]
  \centering
  \includegraphics[width=\linewidth]{images/f\_8.png}
\end{figure}
\begin{figure}[htbp]
  \centering
  \includegraphics[width=\linewidth]{images/f\_20.png}
\end{figure}
\begin{figure}[htbp]
  \centering
  \includegraphics[width=\linewidth]{images/g\_5.png}
\end{figure}
\begin{figure}[htbp]
  \centering
  \includegraphics[width=\linewidth]{images/g\_8.png}
\end{figure}
\begin{figure}[htbp]
  \centering
  \includegraphics[width=\linewidth]{images/g\_20.png}
\end{figure}
\newpage
code:\\
\begin{lstlisting}[language=Python]
import numpy as np
import matplotlib.pyplot as plt
def lagrange(x, points):
    lagrange_results = [];
    for i in range(0, len(points)):
        numerator = 1;
        denominator = 1;
        for j in range(0, len(points)):
            if(i == j):
                continue;
            numerator *= (x - points[j][0]);
            denominator *= (points[i][0] - points[j][0]);
        lagrange_results.append(numerator / denominator);
    result = 0;
    for i in range(0, len(points)):
        result += points[i][1] * lagrange_results[i]
    return result;

def f(x):
    return 1 / (1 + 25 * x**2);
def g(x):
    return (abs(x)) ** (1/2);

def get_x_coords(interval, num_points):
    start = interval[0];
    orig_start = start
    end = interval[1];
    result = [];
    result.append(start);
    for i in range(0, num_points - 1):
        start += (end - orig_start) / (num_points - 1)
        result.append(start)
    return result;
def do_the_thing(my_function, n):
    x_coords = get_x_coords([-1, 1], n);
    actual_function_values = []
    for x in x_coords:
        actual_function_values.append([x, my_function(x)]);

    x_plot = np.linspace(-1, 1, 1000)
    y_actual = []
    y_interp = []
    for i in range(0, len(x_plot)):
        y_actual.append(my_function(x_plot[i]));
        y_interp.append(lagrange(x_plot[i], actual_function_values));
    
    plt.plot(x_plot, y_actual, label=f'Actual Function {my_function.__name__}(x)')
    plt.plot(x_plot, y_interp, '--', label='Lagrange Interpolation')
    plt.scatter(x_coords, [my_function(x) for x in x_coords], color='red', label='Interpolation Nodes')
    plt.title(f'Lagrange Interpolation vs Actual Function, n = {n}')
    plt.xlabel('x')
    plt.ylabel('y')
    plt.legend()
    plt.grid(True)
    plt.show()
def main():
    functions = [f, g]
    nums = [5, 8, 20]
    for function in functions:
        for num in nums:
            do_the_thing(function, num)
if __name__ == "__main__":
    main();
\end{lstlisting}
\end{document}