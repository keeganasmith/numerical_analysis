\documentclass{article}
\usepackage{fancyhdr}
\usepackage{lipsum}  
\usepackage{listings} 
\usepackage{xcolor}   
\usepackage{amsmath}
\usepackage{enumitem}
\usepackage{graphicx}
% Define macros for title and author
\newcommand{\thetitle}{MATH 417 502 \\ Homework 4}
\newcommand{\theauthor}{Keegan Smith}

\title{\thetitle}
\author{\theauthor}

\pagestyle{fancy}
\fancyhf{}  % Clear all header and footer fields
\fancyhead[L]{\nouppercase{\rightmark}}
\fancyhead[C]{\thetitle}  % Title in the center
\fancyhead[R]{\theauthor}  % Your name on the right

\lstset{ %
  backgroundcolor=\color{white},   % choose the background color
  basicstyle=\ttfamily\small,          % size of fonts used for the code
  keywordstyle=\color{black},           % color for keywords
  commentstyle=\color{black},          % color for comments
  stringstyle=\color{black},             % color for strings
  numbers=left,                        % where to put the line-numbers
  numberstyle=\tiny\color{gray},       % style for line-numbers
  stepnumber=1,                        % the step between two line-numbers
  numbersep=5pt,                       % how far the line-numbers are from the code
  frame=single,                        % adds a frame around the code
  rulecolor=\color{black},             % frame color
  breaklines=true,                     % automatic line breaking
  breakatwhitespace=false,             % automatic breaks should only happen at whitespace
  showspaces=false,                    % don't show spaces in the code
  showstringspaces=false,              % don't show spaces in strings
  showtabs=false,                      % don't show tabs in the code
}

\begin{document}
\maketitle
\section*{Problem 1}
\begin{figure}[htbp]
  \centering
  \includegraphics[width=\linewidth]{images/f\_5.png}
\end{figure}

\begin{figure}[htbp]
  \centering
  \includegraphics[width=\linewidth]{images/f\_8.png}
\end{figure}
\begin{figure}[htbp]
  \centering
  \includegraphics[width=\linewidth]{images/f\_20.png}
\end{figure}
\begin{figure}[htbp]
  \centering
  \includegraphics[width=\linewidth]{images/g\_5.png}
\end{figure}
\begin{figure}[htbp]
  \centering
  \includegraphics[width=\linewidth]{images/g\_8.png}
\end{figure}
\begin{figure}[htbp]
  \centering
  \includegraphics[width=\linewidth]{images/g\_20.png}
\end{figure}
\newpage
code:\\
\begin{lstlisting}[language=Python]
import numpy as np
import matplotlib.pyplot as plt
def lagrange(x, points):
    lagrange_results = [];
    for i in range(0, len(points)):
        numerator = 1;
        denominator = 1;
        for j in range(0, len(points)):
            if(i == j):
                continue;
            numerator *= (x - points[j][0]);
            denominator *= (points[i][0] - points[j][0]);
        lagrange_results.append(numerator / denominator);
    result = 0;
    for i in range(0, len(points)):
        result += points[i][1] * lagrange_results[i]
    return result;

def f(x):
    return 1 / (1 + 25 * x**2);
def g(x):
    return (abs(x)) ** (1/2);

def get_x_coords(interval, num_points):
    start = interval[0];
    orig_start = start
    end = interval[1];
    result = [];
    result.append(start);
    for i in range(0, num_points - 1):
        start += (end - orig_start) / (num_points - 1)
        result.append(start)
    return result;
def do_the_thing(my_function, n):
    x_coords = get_x_coords([-1, 1], n);
    actual_function_values = []
    for x in x_coords:
        actual_function_values.append([x, my_function(x)]);

    x_plot = np.linspace(-1, 1, 1000)
    y_actual = []
    y_interp = []
    for i in range(0, len(x_plot)):
        y_actual.append(my_function(x_plot[i]));
        y_interp.append(lagrange(x_plot[i], actual_function_values));
    
    plt.plot(x_plot, y_actual, label=f'Actual Function {my_function.__name__}(x)')
    plt.plot(x_plot, y_interp, '--', label='Lagrange Interpolation')
    plt.scatter(x_coords, [my_function(x) for x in x_coords], color='red', label='Interpolation Nodes')
    plt.title(f'Lagrange Interpolation vs Actual Function, n = {n}')
    plt.xlabel('x')
    plt.ylabel('y')
    plt.legend()
    plt.grid(True)
    plt.show()
def main():
    functions = [f, g]
    nums = [5, 8, 20]
    for function in functions:
        for num in nums:
            do_the_thing(function, num)
if __name__ == "__main__":
    main();
\end{lstlisting}
\section*{Problem 2}
\begin{enumerate}
\item 
Recall the max error is bounded by: \\
\[
|f(x) - p(x)| \leq \frac{(b-a)^{n+1}}{(n+1)!}(\max |f^{n+1}(x)|)
\]
So our error for $f(x) = e^{\lambda x}$ is bounded by: \\
\begin{align*}
\lim_{n \rightarrow \infty}\frac{(b-a)^{n+1}}{(n+1)!}(\max |f^{n+1}(x)|) &= \lim_{n \rightarrow \infty}\frac{(b-a)^{n+1}}{(n+1)!}(\max |\lambda^{n+1}e^{\lambda x}|)
\end{align*}
if $\lambda <0$ then $\max |\lambda^{n+1}e^{\lambda x}| = \lambda ^ {n+1}e^{\lambda a}$. If $\lambda > 0$ then  $max |\lambda^{n+1}e^{\lambda x}| = \lambda ^ {n+1}e^{\lambda b}$. Either way, the max is $\lambda^{n+1}$ multiplied by some constant $C$. Thus we have:\\
\begin{align*}
\lim_{n \rightarrow \infty}\frac{(b-a)^{n+1}}{(n+1)!}(\max |\lambda^{n+1}e^{\lambda x}|) &= \lim_{n \rightarrow \infty}\frac{(b-a)^{n+1}}{(n+1)!}\lambda^{n+1} \cdot C\\
\end{align*}
\begin{align*}
&= \lim_{n \rightarrow \infty}\frac{(b-a)\lambda(b-a)\lambda(b-a)\lambda \cdot ... \cdot (b-a)\lambda}{(n+1)(n)(n-1) \cdot ... \cdot (1)}\cdot C\\
&=  \lim_{n \rightarrow \infty}\frac{(b-a)\lambda}{n+1} \cdot \frac{(b-a)\lambda}{n} \cdot \frac{(b-a) \lambda}{n-1} \cdot ... \cdot \frac{(b-a)\lambda}{1} \cdot C\\
&= 0 \cdot 0 \cdot 0 \cdot ... \cdot \frac{(b-a)\lambda}{1} \cdot C\\
&= 0
\end{align*}
\item The same does not hold true for $f(x) = 4(1+x^2)^{-1}$:\\
\begin{align*}
f'(x) &= -4(2)x(1+x^2)^{-2} \\
f''(x) &= -8(1+x^2)^{-2} + -4(2)x(-2)(2x)(1 + x^2)^{-3} \\
|f'''(x)| &\geq 4(2x)(2x)(2x)(1)(2)(3)(1+x^2)^{-4}\\
|f^{n+1}(x)| &\geq 4(2^{n+1}x^{n+1})n! \cdot (1 + x^2)^{-(n+1)}
\end{align*}
We will say that $f^{n+1}(x)$ is it's maximum at $f^{n+1}(C)$ ($C$ is some constant on the interval $[a,b]$). \\
Plugging this in we get:\\
\begin{align*}
\lim_{n \rightarrow \infty}\frac{(b-a)^{n+1}}{(n+1)!}(\max |f^{n+1}(x)|) &= \lim_{n \rightarrow \infty}\frac{(b-a)^{n+1}}{(n+1)!}(4(2^{n+1}C^{n+1})n! \cdot (1 + C^2)^{-(n+1)})\\
&= \lim_{n \rightarrow \infty}\frac{(b-a)^{n+1}}{(n+1)!}4(2^{n+1})n! \cdot \frac{C^{n+1}}{(1 + C^2)^{n+1}}\\
&= \lim_{n \rightarrow \infty}\frac{(b-a)^{n+1}}{(n+1)!}4(2^{n+1})n! \cdot (\frac{C}{1 + C^2})^{n+1}\\
&=\lim_{n \rightarrow \infty}\frac{n!}{(n+1)!}4(2^{n+1})(b-a)^{n+1} \cdot  (\frac{C}{1 + C^2})^{n+1}\\
&=  \lim_{n \rightarrow \infty}\frac{1}{(n+1)}4(2^{n+1})(b-a)^{n+1} \cdot  (\frac{C}{1 + C^2})^{n+1}\\
&=  \lim_{n \rightarrow \infty}\frac{4(2(b-a))^{n+1}}{(n+1)} \cdot  (\frac{C}{1 + C^2})^{n+1}\\
&= \lim_{n \rightarrow \infty}\frac{4}{(n+1)} \cdot  (\frac{C \cdot 2(b-a)}{1 + C^2} )^{n+1}
\end{align*}
We can see that if $(\frac{C \cdot 2(b-a)}{1 + C^2} ) > 1$, this limit approaches infinity. Thus the limit does not necessarily converge to 0. 

\end{enumerate}
\section*{Problem 3}
\begin{align*}
P_{0, 0} &= 5.3 \\
P_{1, 1} &= 2 \\
P_{2, 2} &= 3.19 \\
P_{3, 3} &= 1 \\
\end{align*}
\begin{align*}
P_{0, 1} &= \frac{(x - (-.1))(2) - (x - 0)(5.3)}{0 - (-.1)} \\
&=  \frac{(x + .1)(2) - (x)(5.3)}{.1} \\
&= \frac{2x + .2 - 5.3x}{.1} \\
&= 20x + 2 - 53x \\
&= -33x + 2
\end{align*}
\begin{align*}
P_{1, 2} &= \frac{(x - 0)(3.19) - (x - .2)(2)}{.2 - 0} \\
&=  \frac{3.19x - 2x + .4}{.2} \\
&= \frac{1.19x + .4}{.2} \\
&=  5.95x + 2\\
\end{align*}
\begin{align*}
P_{2, 3} &= \frac{(x - .2)(1) - (x - .3)(3.19)}{.3 - .2}\\
&= \frac{x - .2 - 3.19x + 0.957}{.1}\\
&= \frac{-2.19x + 0.757}{.1} \\
&= -21.9x + 7.57\\
\end{align*}
\begin{align*}
P_{0, 2} &= \frac{(x - (-.1))(5.95x + 2) - (x - .2)(-33x + 2)}{.2 - (-.1)} \\
&=  \frac{(x + .1)(5.95x + 2) - (x - .2)(-33x + 2)}{.3}\\
&= \frac{5.95x^2 + 2x + .595x + .2 - (-33x^2 + 2x + 6.6x -.4)}{.3} \\
&= \frac{5.95x^2 + 2x + .595x + .2 + 33x^2 - 2x - 6.6x +.4)}{.3} \\
&= \frac{38.95x^2 - 6.005x +.6)}{.3} \\
\end{align*}
\begin{align*}
P_{1, 3} &= \frac{(x - 0)(-21.9x + 7.57) - (x - .3)(5.95x +2)}{.3} \\
&= \frac{(x)(-21.9x + 7.57) - (x - .3)(5.95x +2)}{.3} \\
&= \frac{-21.9x^2 + 7.57x - (5.95x^2 + 2x - 1.785x - .6)}{.3} \\
&= \frac{-21.9x^2 + 7.57x - 5.95x^2 - 2x + 1.785x + .6)}{.3} \\
&= \frac{-27.85x^2 + 7.355x + .6)}{.3} \\
\end{align*}
Final polynomial is:
\begin{align*}
P_{0, 3} &= \frac{(x - (-.1))( \frac{-27.85x^2 + 7.355x + .6)}{.3}) - (x - .3)( \frac{38.95x^2 - 6.005x +.6)}{.3})}{.3 - (-.1)}
\end{align*}

\end{document}