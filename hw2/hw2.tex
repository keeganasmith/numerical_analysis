\documentclass{article}
\usepackage{fancyhdr}
\usepackage{lipsum}  
\usepackage{listings} 
\usepackage{xcolor}   
\usepackage{amsmath}
\usepackage{enumitem}

% Define macros for title and author
\newcommand{\thetitle}{MATH 417 502 \\ Homework 2}
\newcommand{\theauthor}{Keegan Smith}

\title{\thetitle}
\author{\theauthor}

\pagestyle{fancy}
\fancyhf{}  % Clear all header and footer fields
\fancyhead[L]{\nouppercase{\rightmark}}
\fancyhead[C]{\thetitle}  % Title in the center
\fancyhead[R]{\theauthor}  % Your name on the right

\lstset{ %
  backgroundcolor=\color{lightgray},   % choose the background color
  basicstyle=\ttfamily\small,          % size of fonts used for the code
  keywordstyle=\color{blue},           % color for keywords
  commentstyle=\color{green},          % color for comments
  stringstyle=\color{red},             % color for strings
  numbers=left,                        % where to put the line-numbers
  numberstyle=\tiny\color{gray},       % style for line-numbers
  stepnumber=1,                        % the step between two line-numbers
  numbersep=5pt,                       % how far the line-numbers are from the code
  frame=single,                        % adds a frame around the code
  rulecolor=\color{black},             % frame color
  breaklines=true,                     % automatic line breaking
  breakatwhitespace=false,             % automatic breaks should only happen at whitespace
  showspaces=false,                    % don't show spaces in the code
  showstringspaces=false,              % don't show spaces in strings
  showtabs=false,                      % don't show tabs in the code
}

\begin{document}
\maketitle
\section*{Problem 1}
\begin{enumerate}[label=\alph*.)]
\item The first 20 iterations of the Newton method are shown below: \\
\begin{tabular}{|c|c|c|c|}
\hline
Iteration & Approximation & Residual\\
\hline
0 & 1.000000e+00 & 7.182818e-01\\
\hline
1 & 5.819767e-01 & 2.075957e-01\\
\hline
2 & 3.190550e-01 & 5.677201e-02\\
\hline
3 & 1.679962e-01 & 1.493591e-02\\
\hline
4 & 8.634887e-02 & 3.837726e-03\\
\hline
5 & 4.379570e-02 & 9.731870e-04\\
\hline
6 & 2.205769e-02 & 2.450693e-04\\
\hline
7 & 1.106939e-02 & 6.149235e-05\\
\hline
8 & 5.544905e-03 & 1.540144e-05\\
\hline
9 & 2.775014e-03 & 3.853917e-06\\
\hline
10 & 1.388149e-03 & 9.639248e-07\\
\hline
11 & 6.942351e-04 & 2.410369e-07\\
\hline
12 & 3.471577e-04 & 6.026621e-08\\
\hline
13 & 1.735889e-04 & 1.506742e-08\\
\hline
14 & 8.679696e-05 & 3.766965e-09\\
\hline
15 & 4.339911e-05 & 9.417547e-10\\
\hline
16 & 2.169971e-05 & 2.354406e-10\\
\hline
17 & 1.084989e-05 & 5.886025e-11\\
\hline
18 & 5.424958e-06 & 1.471512e-11\\
\hline
19 & 2.712481e-06 & 3.678613e-12\\
\hline
20 & 1.356302e-06 & 9.199308e-13\\
\hline
\end{tabular} \\ 
From the above, we can see that Newton's method converges, but not as quickly as we would like. \\
Requirements to converge quadratically: \\
\[
g'(x*) = 0
\]
\[
g''(x) \leq M
\]
In our case, from newton's method: \\
\[
g(x) = x - \frac{e^x - x - 1}{e^x - 1}
\]
Thus we have:
\begin{align*}
g'(x) &= 1 - ((-e^x)(e^x-1)^{-2}(e^x-x-1) + (e^x - 1)^{-1}(e^x - 1)) \\
g'(0) &= 1 - ((-1)(1-1)^{-2}(1-0 - 1) + \frac{0}{0}) \\
\end{align*}
So $g'(0)$ is undefined, and thus does not satisfy the condition for quadratic convergence. \\
\item Start with $\mu(x)$ and $\mu'(x)$:
\[
\mu(x) = \frac{f(x)}{f'(x)}
\]
\begin{align*}
\mu'(x) &=\frac{d}{dx}(f(x)(f'(x))^{-1} \\
&= \frac{f'(x)}{f'(x)} + (-1)\cdot f''(x) \cdot f'(x)^{-2} \cdot f(x) \\
&= \frac{f'(x)}{f'(x)} - \frac{f''(x) \cdot f(x)}{f'(x)^2}\\
\end{align*}
Plugging into $g(x)$: \\
\begin{align*}
g(x) &= x - \frac{\mu(x)}{\mu'(x)} \\
&= x - \frac{f(x)}{f'(x) \cdot ( \frac{f'(x)}{f'(x)} - \frac{f''(x) \cdot f(x)}{f'(x)^2})}\\
&= x - \frac{f(x)}{\frac{f'(x)^2}{f'(x)} - \frac{f''(x) \cdot f(x)}{f'(x)}} \\
&=  x - \frac{f(x)}{f'(x)^{-1}(f'(x)^2 - f''(x) \cdot f(x))} \\
&= x - \frac{f(x) \cdot f'(x)}{f'(x)^2 - f''(x) \cdot f(x)} \\
\end{align*}
\item We can see that the new $g(x)$ is approximately correct within the first 5 iterations: \\
\begin{tabular}{|c|c|c|c|}
\hline
Iteration & Approximation & Residual\\
\hline
0 & 1.000000e+00 & 7.182818e-01\\
\hline
1 & -2.342106e-01 & 2.540578e-02\\
\hline
2 & -8.458280e-03 & 3.567061e-05\\
\hline
3 & -1.189018e-05 & 7.068790e-11\\
\hline
4 & -4.218591e-11 & 0.000000e+00\\
\hline
5 & -4.218591e-11 & 0.000000e+00\\
\hline
\end{tabular} \\
This $g(x)$ obviously converges much more quickly than the original, and thus it can be concluded that this $g(x)$ converges quadratically. 
\end{enumerate}
\section*{Problem 2}
\begin{enumerate}
\item We know that if $|g'(x*)| < 1$ then $g(x)$ converges. So for $a$ we have:\\
\begin{align*}
g(x) &= \frac{1}{21}(20x + 21x^{-2})\\
g'(x) &= \frac{1}{21}(20 - 42x^{-3}) \\
g'(21^{\frac{1}{3}} &= \frac{1}{21}(20 - 42(21^{\frac{1}{3}})^{-3}) \\
&=  \frac{1}{21}(20 - 42(21^{-1}) \\
&= \frac{1}{21}(20 - 2) \\
&= \frac{1}{21}(18) \\
&= \frac{18}{21} < 1
\end{align*}
Thus $a$ likely converges, but does not converge quadratically since $g'(x) \neq 0$. \\
for $b$ we have: \\
\begin{align*}
g(x) &= x - \frac{1}{3}(x^3 - 21)x^{-2}, \\
g'(x) &= 1 - \frac{1}{3}((3x^2)x^{-2} + (x^3 - 21)(-2)x^{-3}), \\
g'(x) &= 1 - \frac{1}{3}(3 + (x^3 - 21)(-2)x^{-3}), \\
g'(21^{\frac{1}{3}}) &= 1 - \frac{1}{3}(3 + (21 - 21)(-2)(21^{-1})) \\
&= 1 - \frac{1}{3}(3) \\
& = 1 - 1 \\
&= 0 \\
\end{align*}
so $b$ converges, and $b$ also likely quadratically converges since $g'(x*) \equiv 0$\\
For $c$ we have:\\
\begin{align*}
g(x) &= x - (x^4 - 21x)(x^2 - 21)^{-1} \\
g'(x) &= 1 - ((4x^3 - 21)(x^2 - 21)^{-1} + (x^4 - 21x)(-1)(2x)(x^2 - 21)^{-2} \\
g'(21^{\frac{1}{3}})&= 1 - (4(21 - 21)(21^{\frac{2}{3}} - 21)^{-1} - (21^{\frac{4}{3}} - 21^{\frac{4}{3}})(2(21^{\frac{1}{3}}))(21^{\frac{2}{3}} - 21)^{-2}) \\
&= 1 - (0 - 0) \\
&= 1 \\
\end{align*}
$g'(x*) \geq 1$ so $c$ does not converge linearly or quadratically. 
\end{enumerate}
\end{document}
