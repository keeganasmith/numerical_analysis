\documentclass{article}
\usepackage{fancyhdr}
\usepackage{lipsum}  
\usepackage{listings} 
\usepackage{xcolor}   
\usepackage{amsmath}
\usepackage{enumitem}

% Define macros for title and author
\newcommand{\thetitle}{MATH 417 502 \\ Homework 3}
\newcommand{\theauthor}{Keegan Smith}

\title{\thetitle}
\author{\theauthor}

\pagestyle{fancy}
\fancyhf{}  % Clear all header and footer fields
\fancyhead[L]{\nouppercase{\rightmark}}
\fancyhead[C]{\thetitle}  % Title in the center
\fancyhead[R]{\theauthor}  % Your name on the right

\lstset{ %
  backgroundcolor=\color{white},   % choose the background color
  basicstyle=\ttfamily\small,          % size of fonts used for the code
  keywordstyle=\color{black},           % color for keywords
  commentstyle=\color{black},          % color for comments
  stringstyle=\color{black},             % color for strings
  numbers=left,                        % where to put the line-numbers
  numberstyle=\tiny\color{gray},       % style for line-numbers
  stepnumber=1,                        % the step between two line-numbers
  numbersep=5pt,                       % how far the line-numbers are from the code
  frame=single,                        % adds a frame around the code
  rulecolor=\color{black},             % frame color
  breaklines=true,                     % automatic line breaking
  breakatwhitespace=false,             % automatic breaks should only happen at whitespace
  showspaces=false,                    % don't show spaces in the code
  showstringspaces=false,              % don't show spaces in strings
  showtabs=false,                      % don't show tabs in the code
}

\begin{document}
\maketitle
\section*{Problem 1}
\begin{enumerate}[label=\alph*.)]
\item
Our system of equations can be re-written as:\\
\begin{align*}
x_1 + 2x_2 + 3x_3 - \lambda x_1 &= 0 \\
4x_1 + 5x_2 + 6x_3 - \lambda x_2 &= 0 \\
7x_1 + 8x_2 + 10x_3 - \lambda x_3 &= 0 \\
x_1^2 + x_2^2 + x_3^2 - 1 &= 0\\
\end{align*}
The jacobian of this system is:\\
\[
\begin{bmatrix}
1 -\lambda & 2 & 3 & -\lambda x_1 \\
4 & 5 - \lambda & 6 & -x_2 \\
7 & 8 & 10 - \lambda & -x_3 \\
2x_1 & 2x_2 & 2x_3 & 0
\end{bmatrix}
\] 
Thus the Newton iteration looks like: \\
\[
x^{n+1} = x^n - \begin{bmatrix}
1 -\lambda & 2 & 3 & -\lambda x^n_1 \\
4 & 5 - \lambda & 6 & -x^n_2 \\
7 & 8 & 10 - \lambda & -x^n_3 \\
2x^n_1 & 2x^n_2 & 2x^n_3 & 0
\end{bmatrix}^{-1}\begin{bmatrix}
x^n_1 + 2x^n_2 + 3x^n_3 - \lambda x^n_1 \\
4x^n_1 + 5x^n_2 + 6x^n_3 - \lambda x^n_2 \\
7x^n_1 + 8x^n_2 + 10x^n_3 - \lambda x^n_3\\
(x^n_1)^2 + (x^n_2)^2 + (x^n_3)^2 - 1\\
\end{bmatrix}
\]
Essentially we will pick an initial vector $x_0$ and plug this value into our equation above to get the next vector $x^{n+1}$. We continuously do this until we are pretty close to 0. We repeat the process for multiple $x_0$ until we find all of our solutions. \\
\item My program found the following solutions to the system of equations: \\
Solution 0:  [-0.22464024 -0.59734221 -1.06500369 16.65147157]\\
Solution 1:  [ 1.04516341 -0.32819325 -0.41221205 -1.18226152]\\
Solution 2:  [ 0.39884723 -1.03663168  0.58667158  0.12967112]\\
where the first 3 values are eigenvectors $x$ and the last value is the corresponding eigenvalue $\lambda$. Thus we have the eigenvectors:\\ 
\[
\begin{bmatrix}
-0.22464024 \\-0.59734221\\ -1.06500369
\end{bmatrix},
\begin{bmatrix}
 1.04516341 \\ -0.32819325\\ -0.41221205
\end{bmatrix}
\begin{bmatrix}
 0.39884723 \\ -1.03663168 \\ 0.58667158
\end{bmatrix}
\]
And their respective eigenvalues:\\
16.6515, -1.1823, 0.1297 \\
my code to accomplish this is below:\\
\begin{lstlisting}[language=Python]
import numpy as np
import random
from concurrent.futures import ThreadPoolExecutor, as_completed
import threading
import copy
NUM_ITER = 500
global_solutions = []
lock = threading.Lock()
def compute_jacobian_matrix(x_vector):
    my_jacobian = np.array([[0, 2, 3, 0], [4, 0, 6, 0], [7, 8, 0, 0], [0, 0 , 0, 0]])

    my_jacobian[0][0] = 1 - x_vector[3][0]
    my_jacobian[0][3] = -x_vector[0][0]
    my_jacobian[1][1] = 5 - x_vector[3][0]
    my_jacobian[1][3] = -x_vector[1][0]
    my_jacobian[2][2] = 10 - x_vector[3][0]
    my_jacobian[2][3] = -x_vector[2][0]
    my_jacobian[3][0] = 2 * x_vector[0][0]
    my_jacobian[3][1] = 2 * x_vector[1][0]
    my_jacobian[3][2] = 2 * x_vector[2][0]
    return my_jacobian
def compute_f(x_vector):
    my_f = np.array([[0], [0], [0], [0]])

    my_f[0][0] = x_vector[0][0] + 2 * x_vector[1][0] + 3 * x_vector[2][0] - x_vector[3][0] * x_vector[0][0]
    my_f[1][0] = 4 * x_vector[0][0] + 5 * x_vector[1][0] + 6 * x_vector[2][0] - x_vector[3][0] * x_vector[1][0]
    my_f[2][0] = 7 * x_vector[0][0] + 8 * x_vector[1][0] + 10 * x_vector[2][0] - x_vector[3][0] * x_vector[2][0]
    my_f[3][0] = x_vector[0][0]**2 + x_vector[1][0]**2 + x_vector[2][0]**2 - 1
    return my_f
def iterate(initial_x):
    for i in range(0, NUM_ITER):
        jacobian = compute_jacobian_matrix(initial_x)
        my_f = compute_f(initial_x)
        initial_x = initial_x + np.linalg.solve(jacobian, -my_f)
    return initial_x
def perform_iteration(i, start, end):
    #print("got here")
    j = random.uniform(start, end)
    k = random.uniform(start, end)
    l = random.uniform(start, end)
    m = random.uniform(start, end)
    try:
        result = tuple(iterate(np.array([[j], [k], [l], [m]])).flatten())
        
        return result
    except Exception as e:
        return None
if __name__ == "__main__":
    start = -10
    end = 10
    num_attempts = 10000
    with ThreadPoolExecutor() as executor:
        futures = [executor.submit(perform_iteration, i, start, end) for i in range(num_attempts)]
        for future in as_completed(futures):
            result = future.result()
            if(result):
                with lock:
                    global_solutions.append(result)
    while(True):
        result = []
        for i in range(0, 3):
            result.append(np.array(global_solutions[random.randint(0, len(global_solutions))]))
        result = np.array(result)
        eigenvectors = []
        for i in range(0, len(result)):
            eigenvectors.append(result[i][:3])
        determinant = np.linalg.det(eigenvectors)
        if(not (determinant <= 10**(-2) and determinant >= -10**(-2))):
            for i in range(0, len(result)):
                print(f"Solution {i}: ", result[i]) 
            print("determinant was: ", determinant)
            break;
        print("determinant was approx. 0, trying again")
\end{lstlisting}
\end{enumerate}
\end{document}