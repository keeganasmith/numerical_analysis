\documentclass{article}
\usepackage{fancyhdr}
\usepackage{lipsum}  
\usepackage{listings} 
\usepackage{xcolor}   
\usepackage{amsmath}
\usepackage{enumitem}

% Define macros for title and author
\newcommand{\thetitle}{MATH 417 502 \\ Homework 3}
\newcommand{\theauthor}{Keegan Smith}

\title{\thetitle}
\author{\theauthor}

\pagestyle{fancy}
\fancyhf{}  % Clear all header and footer fields
\fancyhead[L]{\nouppercase{\rightmark}}
\fancyhead[C]{\thetitle}  % Title in the center
\fancyhead[R]{\theauthor}  % Your name on the right

\lstset{ %
  backgroundcolor=\color{white},   % choose the background color
  basicstyle=\ttfamily\small,          % size of fonts used for the code
  keywordstyle=\color{black},           % color for keywords
  commentstyle=\color{black},          % color for comments
  stringstyle=\color{black},             % color for strings
  numbers=left,                        % where to put the line-numbers
  numberstyle=\tiny\color{gray},       % style for line-numbers
  stepnumber=1,                        % the step between two line-numbers
  numbersep=5pt,                       % how far the line-numbers are from the code
  frame=single,                        % adds a frame around the code
  rulecolor=\color{black},             % frame color
  breaklines=true,                     % automatic line breaking
  breakatwhitespace=false,             % automatic breaks should only happen at whitespace
  showspaces=false,                    % don't show spaces in the code
  showstringspaces=false,              % don't show spaces in strings
  showtabs=false,                      % don't show tabs in the code
}

\begin{document}
\maketitle
\section*{Problem 1}
\begin{enumerate}[label=\alph*.)]
\item
Our system of equations can be re-written as:\\
\begin{align*}
x_1 + 2x_2 + 3x_3 - \lambda x_1 &= 0 \\
4x_1 + 5x_2 + 6x_3 - \lambda x_2 &= 0 \\
7x_1 + 8x_2 + 10x_3 - \lambda x_3 &= 0 \\
x_1^2 + x_2^2 + x_3^2 - 1 &= 0\\
\end{align*}
The jacobian of this system is:\\
\[
\begin{bmatrix}
1 -\lambda & 2 & 3 & -\lambda x_1 \\
4 & 5 - \lambda & 6 & -x_2 \\
7 & 8 & 10 - \lambda & -x_3 \\
2x_1 & 2x_2 & 2x_3 & 0
\end{bmatrix}
\] 
Thus the Newton iteration looks like: \\
\[
x^{n+1} = x^n - \begin{bmatrix}
1 -\lambda & 2 & 3 & -\lambda x^n_1 \\
4 & 5 - \lambda & 6 & -x^n_2 \\
7 & 8 & 10 - \lambda & -x^n_3 \\
2x^n_1 & 2x^n_2 & 2x^n_3 & 0
\end{bmatrix}^{-1}\begin{bmatrix}
x^n_1 + 2x^n_2 + 3x^n_3 - \lambda x^n_1 \\
4x^n_1 + 5x^n_2 + 6x^n_3 - \lambda x^n_2 \\
7x^n_1 + 8x^n_2 + 10x^n_3 - \lambda x^n_3\\
(x^n_1)^2 + (x^n_2)^2 + (x^n_3)^2 - 1\\
\end{bmatrix}
\]
Essentially we will pick an initial vector $x_0$ and plug this value into our equation above to get the next vector $x^{n+1}$. We continuously do this until we are pretty close to 0. We repeat the process for multiple $x_0$ until we find all of our solutions. \\
\item My program found the following solutions to the system of equations: \\
Solution 0:  [-0.22464024 -0.59734221 -1.06500369 16.65147157]\\
Solution 1:  [ 1.04516341 -0.32819325 -0.41221205 -1.18226152]\\
Solution 2:  [ 0.39884723 -1.03663168  0.58667158  0.12967112]\\
where the first 3 values are eigenvectors $x$ and the last value is the corresponding eigenvalue $\lambda$. Thus we have the eigenvectors:\\ 
\[
\begin{bmatrix}
-0.22464024 \\-0.59734221\\ -1.06500369
\end{bmatrix},
\begin{bmatrix}
 1.04516341 \\ -0.32819325\\ -0.41221205
\end{bmatrix}
\begin{bmatrix}
 0.39884723 \\ -1.03663168 \\ 0.58667158
\end{bmatrix}
\]
And their respective eigenvalues:\\
16.6515, -1.1823, 0.1297 \\
my code to accomplish this is below:\\
\begin{lstlisting}[language=Python]
import numpy as np
import random
from concurrent.futures import ThreadPoolExecutor, as_completed
import threading
import copy
NUM_ITER = 500
global_solutions = []
lock = threading.Lock()
def compute_jacobian_matrix(x_vector):
    my_jacobian = np.array([[0, 2, 3, 0], [4, 0, 6, 0], [7, 8, 0, 0], [0, 0 , 0, 0]])

    my_jacobian[0][0] = 1 - x_vector[3][0]
    my_jacobian[0][3] = -x_vector[0][0]
    my_jacobian[1][1] = 5 - x_vector[3][0]
    my_jacobian[1][3] = -x_vector[1][0]
    my_jacobian[2][2] = 10 - x_vector[3][0]
    my_jacobian[2][3] = -x_vector[2][0]
    my_jacobian[3][0] = 2 * x_vector[0][0]
    my_jacobian[3][1] = 2 * x_vector[1][0]
    my_jacobian[3][2] = 2 * x_vector[2][0]
    return my_jacobian
def compute_f(x_vector):
    my_f = np.array([[0], [0], [0], [0]])

    my_f[0][0] = x_vector[0][0] + 2 * x_vector[1][0] + 3 * x_vector[2][0] - x_vector[3][0] * x_vector[0][0]
    my_f[1][0] = 4 * x_vector[0][0] + 5 * x_vector[1][0] + 6 * x_vector[2][0] - x_vector[3][0] * x_vector[1][0]
    my_f[2][0] = 7 * x_vector[0][0] + 8 * x_vector[1][0] + 10 * x_vector[2][0] - x_vector[3][0] * x_vector[2][0]
    my_f[3][0] = x_vector[0][0]**2 + x_vector[1][0]**2 + x_vector[2][0]**2 - 1
    return my_f
def iterate(initial_x):
    for i in range(0, NUM_ITER):
        jacobian = compute_jacobian_matrix(initial_x)
        my_f = compute_f(initial_x)
        initial_x = initial_x + np.linalg.solve(jacobian, -my_f)
    return initial_x
def perform_iteration(i, start, end):
    #print("got here")
    j = random.uniform(start, end)
    k = random.uniform(start, end)
    l = random.uniform(start, end)
    m = random.uniform(start, end)
    try:
        result = tuple(iterate(np.array([[j], [k], [l], [m]])).flatten())
        
        return result
    except Exception as e:
        return None
if __name__ == "__main__":
    start = -10
    end = 10
    num_attempts = 10000
    with ThreadPoolExecutor() as executor:
        futures = [executor.submit(perform_iteration, i, start, end) for i in range(num_attempts)]
        for future in as_completed(futures):
            result = future.result()
            if(result):
                with lock:
                    global_solutions.append(result)
    while(True):
        result = []
        for i in range(0, 3):
            result.append(np.array(global_solutions[random.randint(0, len(global_solutions))]))
        result = np.array(result)
        eigenvectors = []
        for i in range(0, len(result)):
            eigenvectors.append(result[i][:3])
        determinant = np.linalg.det(eigenvectors)
        if(not (determinant <= 10**(-2) and determinant >= -10**(-2))):
            for i in range(0, len(result)):
                print(f"Solution {i}: ", result[i]) 
            print("determinant was: ", determinant)
            break;
        print("determinant was approx. 0, trying again")
\end{lstlisting} 
\end{enumerate} 
\section*{Problem 2}
To find the minimum of of $f(x)$ we will need to find all of the potential local minima, e.g where $\nabla f(x) = 0$. Thus we will analytically solve for $\nabla f(x)$ and then use newton iteration to solve for $\nabla f(x) = 0$. We will pick the local minima which makes $f(x)$ the smallest.\\
\[
\nabla f(x) = \begin{bmatrix}
x_1^3 + x_2x_3 - 2x_1x_2 \\
x_2 + x_1x_3 - x_1^2 \\
x_3 + x_1x_2
\end{bmatrix}  =0 \] 
To use Newton's method, we will need to calculate the Jacobian of $\nabla f(x)$: \\
\[
J f(x) = \begin{bmatrix}
3x_1^2 - 2x_2 & x_3 - 2x_1 & x_2 \\
x_3 - 2x_1 & 1 & x_1 \\
x_2 & x_1 & 1
\end{bmatrix} \\
\]
Recall Newton's scheme: \\
\begin{align*}
x^{n+1} &= x^n - Jf(x^n)^-1f(x^n) \\
Jf(x^n)(x^{n+1} - x^n) = -f(x^n) \\
\end{align*}
Implemented in python: \\
\begin{lstlisting}[language=Python]
import numpy as np
import random
from multiprocessing import Process, Manager
NUM_ITER = 500
def compute_jacobian_matrix(x_vector):
    my_jacobian = np.array([[0, 0, 0], [0, 1, 0], [0, 0, 1]])
    x_1 = x_vector[0][0]
    x_2 = x_vector[1][0]
    x_3 = x_vector[2][0]
    my_jacobian[0][0] = 3 * x_1**2 - 2*x_2
    my_jacobian[0][1] = x_3 - 2 * x_1 
    my_jacobian[0][2] = x_2
    my_jacobian[1][0] = x_3 - 2 * x_1
    my_jacobian[1][2] = x_1
    my_jacobian[2][0] = x_2
    my_jacobian[2][1] = x_1
    return my_jacobian
def compute_f(x_vector):
    my_f = np.array([[0], [0], [0]])
    x_1 = x_vector[0][0]
    x_2 = x_vector[1][0]
    x_3 = x_vector[2][0]
    my_f[0][0] = x_1**3 + x_2 * x_3 - 2 * x_1 * x_2
    my_f[1][0] = x_2 + x_1 * x_3 - x_1**2
    my_f[2][0] = x_3 + x_1 * x_2
    return my_f
def iterate(initial_x):
    for i in range(0, NUM_ITER):
        jacobian = compute_jacobian_matrix(initial_x)
        my_f = compute_f(initial_x)
        initial_x = initial_x + np.linalg.solve(jacobian, -my_f)
    return initial_x
def perform_iteration(i, start, end, batch_size, lock, global_solutions):
    results = []
    for i in range(0, batch_size):
        j = random.uniform(start, end)
        k = random.uniform(start, end)
        l = random.uniform(start, end)
        try:
            result = tuple(iterate(np.array([[j], [k], [l]])).flatten())
            
            results.append(result)
        except Exception as e:
            continue
    with lock:
        global_solutions.extend(results)
def f(x_tuple):
    x_1 = x_tuple[0]
    x_2 = x_tuple[1]
    x_3 = x_tuple[2]
    return 1/4 * x_1**4 + 1/2 * x_2**2 + 1/2 * x_3**2 + x_1*x_2*x_3 - (x_1)**2 * x_2
if __name__ == "__main__":
    start = -1
    end = 2
    num_cores = 192
    batch_size = 20000
    num_attempts = num_cores
    processes = []
    manager = Manager()
    global_solutions = manager.list()
    lock = manager.Lock()
    for i in range(0, num_cores):
        p = Process(target = perform_iteration, args=(i, start, end, batch_size, lock, global_solutions))
        p.start()
        processes.append(p)
    for process in processes:
        process.join()
    smallest = 999999999
    smallest_sol = None
    for solution in global_solutions:
        result = f(solution)
        if(result < smallest):
            smallest_sol = solution
            smallest = result
    
    print(f"smallest value obtained was {smallest} at {smallest_sol}")
\end{lstlisting}
I ran this on the LAUNCH cluster with 192 cores and achieved a minimum of -3.0595292600760775 with vector: \\
\[\begin{bmatrix}
1.7945995784244926 \\
1.9367735174859102 \\
-0.3935949441846148 \\
\end{bmatrix}
\]
\section*{Problem 3}
\begin{enumerate}[label=\alph*.)]
\item The distance from a point $x$ from $\hat{x}$ can be given by: \\
\begin{align*}
f(x) &= \sqrt{(x_1 -  0)^2 + (x_2 - \frac{1}{2})^2}\\
&= \sqrt{x_1^2 + (x_2 - \frac{1}{2})^2}
\end{align*}
Since we are just minimizing $f(x)$, and $f(x)$ is non-negative, finding an $x$ which minimizes $f(x)^2$ will be equivalent to the $x$ which minimizes $f(x)$. Thus for simplicity we will write:\\
\[
f(x) = x_1^2 + (x_2 - \frac{1}{2})^2
\]
We are minimizing $f(x)$ with constraints, thus we will use lagrange multipliers:\\
\begin{align*}
\nabla f(x) - \lambda \nabla g(x) &= 0 \\
g(x) &= 0
\end{align*}
We calculate $\nabla f(x)$: \\
\begin{align*}
\nabla f(x) &= \begin{bmatrix}
2x_1 \\
2x_2 - 1 \\
\end{bmatrix}
\end{align*}
$\nabla g(x)$: \\
\begin{align*}
\nabla g(x) &= \begin{bmatrix}
3x_1^2 - 1 \\
-2x_2
\end{bmatrix}
\end{align*}
plugging these into the lagrange system we had above, we get: \\
\[
\begin{bmatrix}
2x_1 - \lambda(3x_1^2 - 1) \\
2x_2 - 1 - \lambda(-2x_2) \\
x_1^3 - x_1 + \frac{1}{2} - x_2^2 \\
\end{bmatrix} = 0
\]
three equations, three unknowns. \\
\item Now we need to setup the newton iteration. We will start by finding the jacobian of the above system: \\
\begin{align*}
J &= \begin{bmatrix}
2 - 6\lambda x_1 & 0 & -3x_1^2 + 1 \\
0 & 2 + 2\lambda & 2x_2 \\
3x_1^2 - 1 & -2x_2 & 0 \\
\end{bmatrix}
\end{align*}
Thus newton's method looks like:\\
\[
x_{n+1} = x_{n} -  \begin{bmatrix}
2 - 6\lambda x_{1n} & 0 & -3x_{n1}^2 + 1 \\
0 & 2 + 2\lambda & 2x_{n2} \\
3x_{n1}^2 - 1 & -2x_{n2} & 0 \\
\end{bmatrix}^{-1}
\begin{bmatrix}
2x_{n1} - \lambda(3x_{n1}^2 - 1) \\
2x_{n2} - 1 - \lambda(-2x_{n2}) \\
x_{n1}^3 - x_{n1} + \frac{1}{2} - x_{n2}^2 \\
\end{bmatrix}
\]

\item I ran this on the cluster and obtained 1.5002859866858482 at vector: \\
\[
\begin{bmatrix}
-0.027279807150974 \\
-1.0000379508424895 \\
-1.0101993230816462 \\
\end{bmatrix}
\]
Code is below: \\
\begin{lstlisting}[language=Python]
import numpy as np
import random
from multiprocessing import Process, Manager
NUM_ITER = 750
vector_ranges = [
    [-1, 1],
    [-2, 0],
    [-2, 0]
]
def compute_jacobian_matrix(x_vector):
    my_jacobian = np.array([[0, 0, 0], [0, 0, 0], [0, 0, 0]])
    x_1 = x_vector[0][0]
    x_2 = x_vector[1][0]
    x_3 = x_vector[2][0]
    my_jacobian[0][0] = 2 - 6 * x_1 * x_3
    my_jacobian[0][2] = -3 * x_2**2 + 1
    my_jacobian[1][1] = 2 + 2 * x_3
    my_jacobian[1][2] = 2 * x_2
    my_jacobian[2][0] = 3 * x_1**2 - 1
    my_jacobian[2][1] = -2 * x_2
    return my_jacobian
def compute_f(x_vector):
    my_f = np.array([[0], [0], [0]])
    x_1 = x_vector[0][0]
    x_2 = x_vector[1][0]
    x_3 = x_vector[2][0]
    my_f[0][0] = x_1**3 + x_2 * x_3 - 2 * x_1 * x_2
    my_f[1][0] = x_2 + x_1 * x_3 - x_1**2
    my_f[2][0] = x_3 + x_1 * x_2
    return my_f
def iterate(initial_x):
    for i in range(0, NUM_ITER):
        jacobian = compute_jacobian_matrix(initial_x)
        my_f = compute_f(initial_x)
        initial_x = initial_x + np.linalg.solve(jacobian, -my_f)
    return initial_x
def perform_iteration(i, start, end, batch_size, lock, global_solutions):
    results = []
    for i in range(0, batch_size):
        j = random.uniform(vector_ranges[0][0], vector_ranges[0][1])
        k = random.uniform(vector_ranges[1][0], vector_ranges[1][1])
        l = random.uniform(vector_ranges[2][0], vector_ranges[2][1])
        try:
            result = tuple(iterate(np.array([[j], [k], [l]])).flatten())
            
            results.append(result)
        except Exception as e:
            continue
    with lock:
        global_solutions.extend(results)
def f(x_tuple):
    x_1 = x_tuple[0]
    x_2 = x_tuple[1]
    return (x_1**2 + (x_2 - 1/2)**2)**(1/2)
if __name__ == "__main__":
    start = -100
    end = 100
    num_cores = 192
    batch_size = 30000
    num_attempts = num_cores
    processes = []
    manager = Manager()
    global_solutions = manager.list()
    lock = manager.Lock()
    for i in range(0, num_cores):
        p = Process(target = perform_iteration, args=(i, start, end, batch_size, lock, global_solutions))
        p.start()
        processes.append(p)
    for process in processes:
        process.join()
    smallest = 999999999
    smallest_sol = None
    for solution in global_solutions:
        result = f(solution)
        if(result < smallest):
            smallest_sol = solution
            smallest = result
    
    print(f"smallest value obtained was {smallest} at {smallest_sol}")
\end{lstlisting}
\end{enumerate}
\end{document}