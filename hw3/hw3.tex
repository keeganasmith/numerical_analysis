\documentclass{article}
\usepackage{fancyhdr}
\usepackage{lipsum}  
\usepackage{listings} 
\usepackage{xcolor}   
\usepackage{amsmath}
\usepackage{enumitem}

% Define macros for title and author
\newcommand{\thetitle}{MATH 417 502 \\ Homework 3}
\newcommand{\theauthor}{Keegan Smith}

\title{\thetitle}
\author{\theauthor}

\pagestyle{fancy}
\fancyhf{}  % Clear all header and footer fields
\fancyhead[L]{\nouppercase{\rightmark}}
\fancyhead[C]{\thetitle}  % Title in the center
\fancyhead[R]{\theauthor}  % Your name on the right

\lstset{ %
  backgroundcolor=\color{white},   % choose the background color
  basicstyle=\ttfamily\small,          % size of fonts used for the code
  keywordstyle=\color{black},           % color for keywords
  commentstyle=\color{black},          % color for comments
  stringstyle=\color{black},             % color for strings
  numbers=left,                        % where to put the line-numbers
  numberstyle=\tiny\color{gray},       % style for line-numbers
  stepnumber=1,                        % the step between two line-numbers
  numbersep=5pt,                       % how far the line-numbers are from the code
  frame=single,                        % adds a frame around the code
  rulecolor=\color{black},             % frame color
  breaklines=true,                     % automatic line breaking
  breakatwhitespace=false,             % automatic breaks should only happen at whitespace
  showspaces=false,                    % don't show spaces in the code
  showstringspaces=false,              % don't show spaces in strings
  showtabs=false,                      % don't show tabs in the code
}

\begin{document}
\maketitle
\section*{Problem 1}
Our system of equations can be re-written as:\\
\begin{align*}
x_1 + 2x_2 + 3x_3 - \lambda x_1 &= 0 \\
4x_1 + 5x_2 + 6x_3 - \lambda x_2 &= 0 \\
7x_1 + 8x_2 + 10x_3 - \lambda x_3 &= 0 \\
x_1^2 + x_2^2 + x_3^2 - 1 &= 0\\
\end{align*}
The jacobian of this system is:\\
\[
\begin{bmatrix}
1 -\lambda & 2 & 3 & -\lambda x_1 \\
4 & 5 - \lambda & 6 & -x_2 \\
7 & 8 & 10 - \lambda & -x_3 \\
2x_1 & 2x_2 & 2x_3 & 0
\end{bmatrix}
\] 
Thus the Newton iteration looks like: \\
\[
x^{n+1} = x^n - \begin{bmatrix}
1 -\lambda & 2 & 3 & -\lambda x^n_1 \\
4 & 5 - \lambda & 6 & -x^n_2 \\
7 & 8 & 10 - \lambda & -x^n_3 \\
2x^n_1 & 2x^n_2 & 2x^n_3 & 0
\end{bmatrix}^{-1}\begin{bmatrix}
x^n_1 + 2x^n_2 + 3x^n_3 - \lambda x^n_1 \\
4x^n_1 + 5x^n_2 + 6x^n_3 - \lambda x^n_2 \\
7x^n_1 + 8x^n_2 + 10x^n_3 - \lambda x^n_3\\
(x^n_1)^2 + (x^n_2)^2 + (x^n_3)^2 - 1\\
\end{bmatrix}
\]
\end{document}