\documentclass{article}
\usepackage{fancyhdr}
\usepackage{lipsum}  
\usepackage{listings} 
\usepackage{xcolor}   
\usepackage{amsmath}
\usepackage{enumitem}
\usepackage[english]{babel}

% Set page size and margins
% Replace `letterpaper' with `a4paper' for UK/EU standard size
\usepackage[letterpaper,top=2cm,bottom=2cm,left=3cm,right=3cm,marginparwidth=1.75cm]{geometry}

% Useful packages
\usepackage{amsmath}
\usepackage{graphicx}
\usepackage[colorlinks=true, allcolors=blue]{hyperref}
% Define macros for title and author
\newcommand{\thetitle}{MATH 417 502 \\ Homework 10}
\newcommand{\theauthor}{Keegan Smith}

\title{\thetitle}
\author{\theauthor}

\pagestyle{fancy}
\fancyhf{}  % Clear all header and footer fields
\fancyhead[L]{\nouppercase{\rightmark}}
\fancyhead[C]{\thetitle}  % Title in the center
\fancyhead[R]{\theauthor}  % Your name on the right
\lstset{ %
  backgroundcolor=\color{white},   % choose the background color
  basicstyle=\ttfamily\small,          % size of fonts used for the code
  keywordstyle=\color{black},           % color for keywords
  commentstyle=\color{black},          % color for comments
  stringstyle=\color{black},             % color for strings
  numbers=left,                        % where to put the line-numbers
  numberstyle=\tiny\color{gray},       % style for line-numbers
  stepnumber=1,                        % the step between two line-numbers
  numbersep=5pt,                       % how far the line-numbers are from the code
  frame=single,                        % adds a frame around the code
  rulecolor=\color{black},             % frame color
  breaklines=true,                     % automatic line breaking
  breakatwhitespace=false,             % automatic breaks should only happen at whitespace
  showspaces=false,                    % don't show spaces in the code
  showstringspaces=false,              % don't show spaces in strings
  showtabs=false,                      % don't show tabs in the code
}
\begin{document}
\maketitle
\section* {Problem 1}
\begin{enumerate}
\item implementation: \\
\begin{lstlisting}
def back_sub(A, b):
    i = len(A) -1 
    coefficients = [0] * len(b)
    while(i >= 0):
        row_sum = 0
        j = len(A) - 1
        while(j > i):
            row_sum += A[i][j] * coefficients[j]
            j -= 1
        b[i][0] -= row_sum
        coefficients[i] = b[i][0] / A[i][i]
        i -= 1
    return coefficients
def forward_sub(A, b):
    coefficients = [0] * len(A)
    for i in range(0, len(A)):
        sum_row = 0
        for j in range(0, i ):
            sum_row += A[i][j] * coefficients[j]
        result = b[i] - sum_row
        coefficients[i] = result / A[i][i]
    return coefficients
def inner_product(a, b):
    result = 0
    for i in range(0, len(a)):
        result += a[i] * b[i]
    return result
def multiply_by_scalar(a, scalar):
    result = []
    for i in range(0, len(a)):
        result.append(a[i] * scalar)
    return result
def add_vectors(a, b):
    result = []
    for i in range(0, len(a)):
        result.append(a[i] + b[i])
    return result
def norm(a):
    result = 0
    for value in a:
        result += value**2
    return result ** .5
def subtract_vectors(a, b):
    return add_vectors(a, multiply_by_scalar(b, -1))
def gram_sum(i, v, theta, m):
    result = [0] * m
    for k in range(0, i):
        inner = inner_product(v[i], theta[k])
        intermed = multiply_by_scalar(theta[k], inner)
        result = add_vectors(intermed, result)
    return result
def transpose(A):
    result = []
    for i in range(0, len(A[0])):
        result.append([0] * len(A))
    for i in range(0, len(A)):
        for j in range(0, len(A[0])):
            result[j][i] = A[i][j]
    return result
def gram_schmidt(A):
    w = []
    v = []
    theta = []
    v = transpose(A)    
    m = len(v[0])
    for i in range(0, len(v)):
        print(theta)
        summation = gram_sum(i, v, theta, m)
        w.append(subtract_vectors(v[i], summation))
        print(w)
        w_norm = norm(w[i])
        theta.append(multiply_by_scalar(w[i], 1/w_norm))
    Q = transpose(theta)
    R = []
    for i in range(0, len(v)):
        R.append([0] * len(v))
        R[i][i] = norm(w[i])
        for j in range(i+1, len(v)):
            R[i][j] = inner_product(v[j], theta[i])
    
    return Q, R
def to_string(A):
    result = ""
    for i in range(0, len(A)):
        for j in range(0, len(A[0])):
            result += str(A[i][j]) + " "
        result += "\n"
    return result
def row_col_product(A, B, i, j):
    result = 0
    for k in range(0, len(A[i])):
        
        result += A[i][k] * B[k][j]
    return result
def multiply(A, B):
    result = []
    for i in range(0, len(A)):
        result.append([0] * len(B[0]))
    for i in range(0, len(A)):
        for j in range(0, len(B[0])):
            result[i][j] = row_col_product(A, B, i, j)
    return result
def convert_matrix_to_latex(A):
    result = "\\begin{bmatrix}\n"
    for i in range(0, len(A)):
        for j in range(0, len(A[i])):
            result += str(A[i][j]) + " & "
        result += "\\\\\n"
    result += "\\end{bmatrix}\n"
    return result
def solve(Q, R, b):
    Q_transpose = transpose(Q)
    Q_transpose_b = multiply(Q_transpose, b)
    result = back_sub(R, Q_transpose_b)
    return result
def main():
    A = [
        [1, 1, 2, 3],
        [2, 2, 2, 2], 
        [4, 3, 2, 2],
        [1, 1, 2, 3],
        [3, 1, 2, 3]
    ]
    
    Q, R = gram_schmidt(A)
    print("Q:")
    print(to_string(Q))
    print("R:")
    print(to_string(R))
    print("QR (should be equivalent to A):")
    mult_result = multiply(Q, R)
    print(to_string(mult_result))
    print("Q in latex:")
    print(convert_matrix_to_latex(Q))
    print("R in latex:")
    print(convert_matrix_to_latex(R))
    B = [[2], [5], [7], [2], [3]]
    result = solve(Q, R, B)
    print(result)
if __name__ == "__main__":
    main()
\end{lstlisting}
\item Q: \\
\[
\begin{bmatrix}
0.1796053020267749 & 0.24217973984824143 & 0.5664411840370205 & 0.29704426289300906 & \\
0.3592106040535498 & 0.48435947969648285 & 0.0944068640061698 & -0.7921180343813354 & \\
0.7184212081070996 & 0.2179617658634174 & -0.5286784384345531 & 0.39605901719066944 & \\
0.1796053020267749 & 0.24217973984824143 & 0.5664411840370205 & 0.29704426289300906 & \\
0.5388159060803247 & -0.7749751675143725 & 0.26433921921727643 & -0.19802950859533192 & \\
\end{bmatrix}
\]
\\
R:
\[
\begin{bmatrix}
5.5677643628300215 & 3.7717113425622726 & 3.951316644589048 & 4.849343154722922 & \\
0 & 1.3319885691653277 & 0.8234111154840213 & 0.5327954276661315 & \\
0 & 0 & 1.9259000257258707 & 3.3231216130171854 & \\
0 & 0 & 0 & 0.39605901719066977 & \\
\end{bmatrix}
\]
Solving for $y$ yields: \\
\[
\begin{bmatrix}
\frac{1}{2} \\
1 \\
2.5 \\
-1.5
\end{bmatrix}
\]
\end{enumerate}
\section*{Problem 2}
Following the gram schmidt algorithm: \\
\begin{align*}
w_{0} &= \begin{bmatrix}1 \\ 1 \\ 2 \end{bmatrix} - 0 \\
&= \begin{bmatrix}1 \\ 1 \\ 2 \end{bmatrix} \\
\theta_{0} &= \frac{1}{\sqrt{6}}  \begin{bmatrix}1 \\ 1 \\ 2 \end{bmatrix} \\
&= \begin{bmatrix}\frac{1}{\sqrt{6}} \\ \frac{1}{\sqrt{6}} \\ \frac{2}{\sqrt{6}} \end{bmatrix}
\end{align*}
\begin{align*}
w_{1} &= \begin{bmatrix}2 \\2 \\1\end{bmatrix} - (\frac{2}{\sqrt{6}} + \frac{2}{\sqrt{6}} + \frac{1}{\sqrt{6}}) \begin{bmatrix}\frac{1}{\sqrt{6}} \\ \frac{1}{\sqrt{6}} \\ \frac{2}{\sqrt{6}} \end{bmatrix} \\
&= \begin{bmatrix}2 - \frac{5}{6} \\ 2 - \frac{5}{6} \\ 2 - \frac{5}{3}\end{bmatrix} \\ 
&= \begin{bmatrix}\frac{7}{6} \\ \frac{7}{6} \\ \frac{1}{3} \end{bmatrix} \\ 
\theta_{1} &= \frac{1}{\sqrt{\frac{17}{6}}} \cdot \begin{bmatrix}\frac{7}{6} \\ \frac{7}{6} \\ \frac{1}{3} \end{bmatrix} \\
&= \frac{6 \sqrt{\frac{17}{6}}}{17} \cdot \begin{bmatrix}\frac{7}{6} \\ \frac{7}{6} \\ \frac{1}{3} \end{bmatrix} \\
&= \begin{bmatrix}\frac{7 \cdot \sqrt{\frac{17}{6}}}{17} \\ \frac{7 \cdot \sqrt{\frac{17}{6}}}{17} \\ \frac{2 \cdot  \sqrt{\frac{17}{6}}}{17} \end{bmatrix} \\
\end{align*}
So we have: \\
\[
Q = \begin{bmatrix}
\frac{1}{\sqrt{6}} & \frac{7 \cdot \sqrt{\frac{17}{6}}}{17} \\ \frac{1}{\sqrt{6}} & \frac{7 \cdot \sqrt{\frac{17}{6}}}{17} \\ \frac{2}{\sqrt{6}} & \frac{2 \cdot  \sqrt{\frac{17}{6}}}{17}
\end{bmatrix}
\]
\[
R = \begin{bmatrix}
\sqrt{6} & \frac{5}{\sqrt{6}} \\
0 & \sqrt{\frac{17}{6}}
\end{bmatrix}
\]
We need to solve: \\
\begin{align*}
Ax &= y \\
QRx &= y \\
Q^TQRx &= Q^Ty \\
Rx &= Q^Ty
\end{align*}
$Q^Ty$ is: \\
\[
\begin{bmatrix}
5.30722777603022\\
2.3094010767585016 \\
\end{bmatrix}
\]
So back substitution yields:\\
\[
x = \begin{bmatrix} \frac{5}{6} \\ \frac{4}{3} \end{bmatrix}
\]
\end{document}
