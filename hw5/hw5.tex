\documentclass{article}
\usepackage{fancyhdr}
\usepackage{lipsum}  
\usepackage{listings} 
\usepackage{xcolor}   
\usepackage{amsmath}
\usepackage{enumitem}
\usepackage{graphicx}
% Define macros for title and author
\newcommand{\thetitle}{MATH 417 502 \\ Homework 4}
\newcommand{\theauthor}{Keegan Smith}

\title{\thetitle}
\author{\theauthor}

\pagestyle{fancy}
\fancyhf{}  % Clear all header and footer fields
\fancyhead[L]{\nouppercase{\rightmark}}
\fancyhead[C]{\thetitle}  % Title in the center
\fancyhead[R]{\theauthor}  % Your name on the right

\lstset{ %
  backgroundcolor=\color{white},   % choose the background color
  basicstyle=\ttfamily\small,          % size of fonts used for the code
  keywordstyle=\color{black},           % color for keywords
  commentstyle=\color{black},          % color for comments
  stringstyle=\color{black},             % color for strings
  numbers=left,                        % where to put the line-numbers
  numberstyle=\tiny\color{gray},       % style for line-numbers
  stepnumber=1,                        % the step between two line-numbers
  numbersep=5pt,                       % how far the line-numbers are from the code
  frame=single,                        % adds a frame around the code
  rulecolor=\color{black},             % frame color
  breaklines=true,                     % automatic line breaking
  breakatwhitespace=false,             % automatic breaks should only happen at whitespace
  showspaces=false,                    % don't show spaces in the code
  showstringspaces=false,              % don't show spaces in strings
  showtabs=false,                      % don't show tabs in the code
}

\begin{document}
\maketitle
\section*{Problem 1}
\begin{figure}[htbp]
  \centering
  \includegraphics[width=\linewidth]{images/f\_5.png}
\end{figure}

\begin{figure}[htbp]
  \centering
  \includegraphics[width=\linewidth]{images/f\_8.png}
\end{figure}
\begin{figure}[htbp]
  \centering
  \includegraphics[width=\linewidth]{images/f\_20.png}
\end{figure}
\begin{figure}[htbp]
  \centering
  \includegraphics[width=\linewidth]{images/g\_5.png}
\end{figure}
\begin{figure}[htbp]
  \centering
  \includegraphics[width=\linewidth]{images/g\_8.png}
\end{figure}
\begin{figure}[htbp]
  \centering
  \includegraphics[width=\linewidth]{images/g\_20.png}
\end{figure}
\newpage
code:\\
\begin{lstlisting}[language=Python]
import numpy as np
import matplotlib.pyplot as plt
import math
def lagrange(x, points):
    lagrange_results = [];
    for i in range(0, len(points)):
        numerator = 1;
        denominator = 1;
        for j in range(0, len(points)):
            if(i == j):
                continue;
            numerator *= (x - points[j][0]);
            denominator *= (points[i][0] - points[j][0]);
        lagrange_results.append(numerator / denominator);
    result = 0;
    for i in range(0, len(points)):
        result += points[i][1] * lagrange_results[i]
    return result;

def f(x):
    return 1 / (1 + 25 * x**2);
def g(x):
    return (abs(x)) ** (1/2);

def compute_cheby_points(n):
    x_coords = [];
    for i in range(0, n):
        x_coords.append(math.cos((2* i + 1) * math.pi / (2*(n - 1) + 2)))
    return x_coords;
def get_x_coords(interval, num_points):
    start = interval[0];
    orig_start = start
    end = interval[1];
    result = compute_cheby_points(num_points)
    
    return result;
def do_the_thing(my_function, n):
    x_coords = get_x_coords([-1, 1], n);
    actual_function_values = []
    for x in x_coords:
        actual_function_values.append([x, my_function(x)]);

    x_plot = np.linspace(-1, 1, 1000)
    y_actual = []
    y_interp = []
    for i in range(0, len(x_plot)):
        y_actual.append(my_function(x_plot[i]));
        y_interp.append(lagrange(x_plot[i], actual_function_values));
    
    plt.plot(x_plot, y_actual, label=f'Actual Function {my_function.__name__}(x)')
    plt.plot(x_plot, y_interp, '--', label='Lagrange Interpolation')
    plt.scatter(x_coords, [my_function(x) for x in x_coords], color='red', label='Interpolation Nodes')
    plt.title(f'Lagrange Interpolation vs Actual Function, n = {n}')
    plt.xlabel('x')
    plt.ylabel('y')
    plt.legend()
    plt.grid(True)
    plt.show()
def main():
    functions = [f, g]
    nums = [5, 8, 20]
    for function in functions:
        for num in nums:
            do_the_thing(function, num)
if __name__ == "__main__":
    main();
\end{lstlisting}
\section*{Problem 2}
From the lecture notes, Simpson's $\frac{3}{8}$'s rule is a result of computing Newton Cotes for $n=3$. Newton Cotes rule gives us a quadrature rule where: \\
\[
w_i = \int_{a}^bL_i(x)dx
\]
for $n$ equidistant points from a to b. We will define $h = \frac{b - a}{n}$, such that $a + h = x_1, a+2h = x_2,$ etc. So for $n = 3$ we have: \\
\begin{align*}
w_0 &= \int_{a}^b\frac{(x-x_1)(x-x_2)(x-x_3)}{(x_0-x_1)(x_0 - x_2)(x_0-x_3)}dx \\
&= \frac{1}{(-h)(-2h)(-3h)}\int_a^b(x^3 - x^2 \cdot x_3 - x^2 \cdot x_2 + x \cdot x_2 \cdot x_3 - x^2 \cdot x_1 + x \cdot x_1 \cdot x_3 + x_1 \cdot x_2 \cdot x - x_1 \cdot x_2 \cdot x_3)dx \\
&=  \frac{1}{(-h)(-2h)(-3h)}\int_a^b(x^3 - x^2(x_3 + x_2 + x_1) + x(x_2 \cdot x_3 + x_1 \cdot x_3 + x_1 \cdot x_2)  - x_1 \cdot x_2 \cdot x_3 )dx \\
&= \frac{1}{(-h)(-2h)(-3h)}(\frac{1}{4}(b^4 - a^4) - (x_3 + x_2 + x_1)(\frac{1}{3}(b^3 - a^3)) + (x_2 \cdot x_3 + x_1 \cdot x_3 + x_1 \cdot x_2)(\frac{1}{2})(b^2 - a^2)\\& \quad - (b - a)( x_1 \cdot x_2 \cdot x_3)) \\
&= \frac{1}{-6h^3} ( \frac{1}{4} ( (a + 3h)^4 - a^4 ) - (3a + 6h) ( \frac{1}{3} ( (a+3h)^3 - a^3 ) )   - (b - a)( x_1 \cdot x_2 \cdot x_3)\\ & \quad + ( (a+2h) \cdot (a+3h) + (a+h) \cdot (a+3h) + (a+h) \cdot (a+2h) ) \cdot \frac{1}{2} ( (a+3h)^2 - a^2 ) ) \\
&= \frac{1}{-6h^3} ( \frac{1}{4} ( a^4+12a^3h+54a^2h^2+108ah^3+81h^4 - a^4 ) - (3a + 6h) ( \frac{1}{3} ( (a+3h)^3 - a^3 ) )\\&\quad - (b - a)( x_1 \cdot x_2 \cdot x_3) \\ & \quad + ( (a+2h) \cdot (a+3h) + (a+h) \cdot (a+3h) + (a+h) \cdot (a+2h) ) \cdot \frac{1}{2} ( (a+3h)^2 - a^2 ) ) \\
&= \frac{1}{-6h^3} ( \frac{1}{4} (12a^3h+54a^2h^2+108ah^3+81h^4 ) - (3a + 6h) ( \frac{1}{3} (9ha^2+27ah^2+27h^3 ) )\\&\quad - (b - a)( x_1 \cdot x_2 \cdot x_3) \\ & \quad + ( (a+2h) \cdot (a+3h) + (a+h) \cdot (a+3h) + (a+h) \cdot (a+2h) ) \cdot \frac{1}{2} ( (a+3h)^2 - a^2 ) ) \\
&= \frac{1}{-6h^3} ( \frac{1}{4} (12a^3h+54a^2h^2+108ah^3+81h^4 ) - (54h^4+9ha^3+27a^2h^2+81ah^3+18h^2a^2)\\&\quad - (b - a)( x_1 \cdot x_2 \cdot x_3) \\ & \quad + ( (a+2h) \cdot (a+3h) + (a+h) \cdot (a+3h) + (a+h) \cdot (a+2h) ) \cdot \frac{1}{2} ( (a+3h)^2 - a^2 ) ) \\
&= \frac{1}{-6h^3} ( (3a^3h+\frac{27}{2}a^2h^2+27ah^3+\frac{81}{4}h^4 ) - (54h^4+9ha^3+27a^2h^2+81ah^3+18h^2a^2)\\ &\quad - (b - a)( x_1 \cdot x_2 \cdot x_3) \\ & \quad + ( (a+2h) \cdot (a+3h) + (a+h) \cdot (a+3h) + (a+h) \cdot (a+2h) ) \cdot \frac{1}{2} ( (a+3h)^2 - a^2 ) ) \\
\end{align*}
\newpage
\begin{align*}
&=  \frac{1}{-6h^3} (-\frac{135}{4}h^4-6ha^3-\frac{27}{2}a^2h^2-54ah^3-18h^2a^2 - (b - a)( x_1 \cdot x_2 \cdot x_3)\\ & \quad + ( (a+2h) \cdot (a+3h) + (a+h) \cdot (a+3h) + (a+h) \cdot (a+2h) ) \cdot \frac{1}{2} ( (a+3h)^2 - a^2 ) ) )\\
&=  \frac{1}{-6h^3} (-\frac{135}{4}h^4-6ha^3-\frac{27}{2}a^2h^2-54ah^3-18h^2a^2 - (b - a)( x_1 \cdot x_2 \cdot x_3)\\ & \quad + ( (a+3h)\cdot (a + 2h + a + h) + (a+h) \cdot (a+2h) ) \cdot \frac{1}{2} ( (a+3h)^2 - a^2 ) ) )\\
&=  \frac{1}{-6h^3} (-\frac{135}{4}h^4-6ha^3-\frac{27}{2}a^2h^2-54ah^3-18h^2a^2 - (b - a)( x_1 \cdot x_2 \cdot x_3)\\ & \quad + ( (a+3h)\cdot (2a + 3h) + (a+h) \cdot (a+2h) ) \cdot \frac{1}{2} ( (a+3h)^2 - a^2 ) )) \\
&=  \frac{1}{-6h^3} (-\frac{135}{4}h^4-6ha^3-\frac{27}{2}a^2h^2-54ah^3-18h^2a^2 - (b - a)( x_1 \cdot x_2 \cdot x_3)\\ & \quad + \frac{27h^2a^2+99h^4}{2}+87ah^3+9ha^3+36a^2h^2) \\
&=  \frac{1}{-6h^3} (-\frac{135}{4}h^4-6ha^3-\frac{27}{2}a^2h^2-54ah^3-18h^2a^2 + \frac{27h^2a^2+99h^4}{2}+87ah^3+9ha^3+36a^2h^2\\& \quad - (b - a)( x_1 \cdot x_2 \cdot x_3)) \\
&= \frac{1}{-6h^3} (-\frac{135h^4}{4}+3ha^3+\frac{99h^4}{2}+33ah^3 - (b - a)( x_1 \cdot x_2 \cdot x_3)) \\
&= \frac{1}{-6h^3} (\frac{63}{4}h^4+3ha^3 + 33ah^3 - (3h)( (a + h) \cdot (a + 2h) \cdot (a + 3h))) \\
&= \frac{1}{-6h^3} (\frac{63}{4}h^4+3ha^3 + 33ah^3 - (3h)(a^3+6ha^2+11ah^2+6h^3)) \\
&= \frac{1}{-6h^2} (\frac{63}{4}h^3 + 3a^3 + 33ah^2 - (3a^3 + 18ha^2 + 33ah^2 + 18h^4)) \\
&= \frac{1}{-6h^2} (\frac{63}{4}h^3 - 18ha^2 - 18h^4) \\
&= \frac{1}{-6h} (\frac{63}{4}h^2 - 18a^2 - 18h^3) \\
\end{align*}
:( \\
\begin{align*}
w_1 &= \int_{a}^b\frac{(x-x_0)(x-x_2)(x-x_3)}{(x_1-x_0)(x_1 - x_2)(x_1-x_3)}dx \\
&= \frac{1}{(h)(-h)(-2h)}\int_{a}^bx^3-x^2x_3-x^2x_2+xx_2x_3-x^2x_0+xx_0x_3+xx_0x_2-x_0x_2x_3 dx \\
&= \frac{1}{2h^3}\int_{a}^b x^3-x^2x_3-x^2x_2+xx_2x_3-x^2x_0+xx_0x_3+xx_0x_2-x_0x_2x_3 dx \\
&= \frac{1}{2h^3}\int_{a}^b x^3-x^2(x_3 + x_2)+x(x_2x_3+x_0x_3+x_0x_2)-x_0x_2x_3 dx \\
&= \frac{1}{2h^3}(\frac{1}{4}(b^4 - a^4) - \frac{1}{3}(x_3 + x_2)(b^3 - a^3) + \frac{1}{2}(b^2 - a^2)(x_2x_3+x_0x_3+x_0x_2) - (b- a)( x_0x_2x_3))\\
&= \frac{1}{2h^3}(\frac{1}{4}((a + 3h)^4 - a^4) - \frac{1}{3}(2a + 5h)((a+3h)^3 - a^3)\\&\quad + \frac{1}{2}((a+3h)^2 - a^2)((a+2h)(a+3h)+a(a+3h)+a(a+2h)) - (3h)( a(a+2h)(a+3h)))\\
&= \frac{1}{2h^3}(-\frac{99h^4}{4}-3ha^3-36ah^3-\frac{39a^2h^2}{2}\\&\quad + \frac{1}{2}((a+3h)^2 - a^2)((a+2h)(a+3h)+a(a+3h)+a(a+2h)) - (3h)( a(a+2h)(a+3h)))\\
&= \frac{1}{2h^3}(-\frac{99h^4}{4}-3ha^3-36ah^3-\frac{39a^2h^2}{2} + \frac{54h^4+57h^2a^2+90ah^3}{2}+6ha^3)\\
&= \frac{1}{2h^3}(\frac{9h^4}{4}+9ah^3+9a^2h^2+3a^3h) \\
&= \frac{1}{2h^2}(\frac{9h^3}{4}+9ah^2+9a^2h+3a^3)
\end{align*}
The weights of Newton Cotes are symmetric, so $w_2 = w_1, w_3 = w_0$. Thus in total we have: \\
\begin{align*}
I(f)  &= w_0 \cdot f(a) + w_1 \cdot f(a + h) + w_2 \cdot f(a + 2h) + w_3 \cdot f(b) \\
&= \frac{1}{-6h} (\frac{63}{4}h^2 - 18a^2 - 18h^3) \cdot (f(a) + f(b))\\&\quad + \frac{1}{2h^2}(\frac{9h^3}{4}+9ah^2+9a^2h+3a^3) \cdot (f(a + h) + f(a + 2h)) \\
&= \frac{b - a}{8}(f(a) + 3f(a + h) + 3f(a + 2h) + f(b)) 
\end{align*}
\section*{Problem 3}
\begin{enumerate}
\item The trapezoidal rule (derived from Newton Cotes with $n=1$): \\
\[
\int_a^b f(x)dx \approx (b-a) \frac{1}{2}(f(a) + f(b))
\]
Thus we have: \\
\begin{align*}
\int_a^bf(x)dx &\approx (1 - 0) \cdot \frac{1}{2}(\frac{1}{1 + 2\cdot 0} + \frac{1}{1 + 2 \cdot 1}) \\
&= \frac{1}{2}(1 + \frac{1}{3}) \\
&= \frac{1}{2}(\frac{4}{3}) \\
&= \frac{2}{3} \\
\end{align*}
\item Simpson's rule (derived from Newton Cotes with $n=2$): \\
\[
\int_a^b f(x)dx \approx  (b-a)\frac{1}{6}(f(a) + 4f(\frac{a + b}{2}) + f(b))
\]
So we have: \\
\begin{align*}
\int_a^b f(x)dx &\approx \frac{1}{6}(1 + 4 \cdot \frac{1}{1 + 2 \cdot \frac{1}{2}} + \frac{1}{3}) \\
&= \frac{1}{6}(1 + 2 + \frac{1}{3}) \\
&= \frac{10}{18} \\
&= \frac{5}{9}
\end{align*}
\item I found that the error dipped below $10^{-6}$ for Newton Cotes with $n=10$ (so 11 evaluations of the integrand are required). \\
This is the code I used: \\
\begin{lstlisting}
from scipy.integrate import newton_cotes
import numpy as np
ACTUAL =  0.5493064 
def f(x):
    return 1/(1 + 2*x)
def n_point_rule(function, interval, n):
    a = interval[0]
    b = interval[1]
    x = np.linspace(a, b, n + 1)
    an, B = newton_cotes(n)
    h = (b - a) / n
    return h * np.sum(an * function(x))
def n3(function, interval): #newton cotes, n=3
    h = (interval[1] - interval[0]) / 3;
    return 3/8 * h * (function(interval[0]) + 3 *function(interval[0] + h) + 3* function(interval[0] + 2*h) + function(interval[1]))
def n4(function, interval):
    h = (interval[1] - interval[0]) / 4;
    a = interval[0]
    return 2/45 * h * (7 * function(a) + 32 * function(a + h) + 12 * function(a + 2 * h) + 32 * function(a + 3* h) + 7 * function(interval[1])) 
def n5(function, interval):
    h = (interval[1] - interval[0]) / 5;
    a = interval[0]
    x_val = a;
    index = 0
    vals =[]
    while(index < 6):
        vals.append(function(x_val))
        x_val += h
        index += 1
    return 5/288 * h * (19* vals[0] + 75 * vals[1] + 50 * vals[2] + 50 * vals[3] + 75 * vals[4] + 19 * vals[5])
def n6(function, interval):
    h = (interval[1] - interval[0]) / 6;
    a = interval[0]
    x_val = a;
    index = 0
    vals =[]
    while(index < 7):
        vals.append(function(x_val))
        x_val += h
        index += 1
    return 1/140 * h * (41* vals[0] + 216 * vals[1] + 27 * vals[2] + 272 * vals[3] + 27 * vals[4] + 216 * vals[5] + 41 * vals[6])
def n7(function, interval):
    h = (interval[1] - interval[0]) / 7;
    a = interval[0]
    x_val = a;
    index = 0
    vals =[]
    while(index < 8):
        vals.append(function(x_val))
        x_val += h
        index += 1
    return 7/17280 * h * (751* vals[0] + 3577 * vals[1] + 1323 * vals[2] + 2989 * vals[3] + 2989 * vals[4] + 1323 * vals[5] + 3577 * vals[6] + 751 * vals[7])
def n8(function, interval):
    h = (interval[1] - interval[0]) / 8;
    a = interval[0]
    x_val = a;
    index = 0
    vals =[]
    while(index < 9):
        vals.append(function(x_val))
        x_val += h
        index += 1
    return 4/14175 * h * (989* vals[0] + 5888 * vals[1] - 928 * vals[2] + 10496 * vals[3] - 4540 * vals[4] + 10496 * vals[5] - 928 * vals[6] + 5888 * vals[7] + 989 * vals[8])
def main():
    interval = [0, 1]
    print("n =3 error: ", abs(n3(f, interval) - ACTUAL))
    print("n=4 error:", abs(n4(f, interval) - ACTUAL))
    print("n=5 error:", abs(n5(f, interval) - ACTUAL))
    print("n=6 error:", abs(n6(f, interval) - ACTUAL))
    print("n=7 error:", abs(n7(f, interval) - ACTUAL))
    print("n=8 error:", abs(n8(f, interval) - ACTUAL))
    for i in range(9, 15):
        print(f"n={i} error:", abs(n_point_rule(f, interval, i)) - ACTUAL)
    
if __name__ == "__main__":
    main()
\end{lstlisting}
\end{enumerate}
\end{document}